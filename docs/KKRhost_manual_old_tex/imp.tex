\documentstyle[11pt]{article}


\textwidth = 16cm
\textheight= 24cm

\hoffset=-0.8cm
\voffset=-2.0cm

\begin{document}

\section{Producing Green's function for impurity calculation}

For impurity calculation you first have to make a file that 
contains information about the positions of atoms which form
the "impurity" cluster. This "impurity" cluster then 
is embedded into the host, and the impurity program recalculates
self-consistently the potential, charge density, etc.  
for the atoms which are inside the impurity cluster, 
assuming the boundary condition with the host. The simplest way 
to make such a file is the following. First make one iteration 
of the 'tb-kkr' program starting from the self-consistent potential for your 
system and use the test-option {\tt clusters} in the 'inputcard'.
You will see the text-format file "clusters" which contains the information 
about tight-binding clusters for all atoms in the inputcard. 
Then you should choose the site, which will be the center site of the impurity 
cluster, cut out the information concerning other atoms, and make the 
"impurity" file (let's call it "scoef") in the format, presented below: 
\bigskip

{\small\tt
  \begin{tabular}{rrrrrr}
    19\phantom{.XXXXXXX} & {}            &       {}     &    {} &  {}  &   {}  
      \\
    0.00000000  &   0.00000000  &   0.00000000 &    3  & 29.0 &   0.000000  \\ 
    0.20412415  &  -0.35355339  &  -0.57735027 &    2  & 29.0 &   0.707107  \\ 
   -0.40824829  &   0.00000000  &  -0.57735027 &    2  & 29.0 &   0.707107  \\
    0.20412415  &   0.35355339  &  -0.57735027 &    2  & 29.0 &   0.707107  \\
    0.00000000  &  -0.70710678  &   0.00000000 &    3  & 29.0 &   0.707107  \\
   -0.61237244  &  -0.35355339  &   0.00000000 &    3  & 29.0 &   0.707107  \\
    0.61237244  &  -0.35355339  &   0.00000000 &    3  & 29.0 &   0.707107  \\
   -0.61237244  &   0.35355339  &   0.00000000 &    3  & 29.0 &   0.707107  \\
    0.61237244  &   0.35355339  &   0.00000000 &    3  & 29.0 &   0.707107  \\
    0.00000000  &   0.70710678  &   0.00000000 &    3  & 29.0 &   0.707107  \\
   -0.20412415  &  -0.35355339  &   0.57735027 &    4  &  0.0 &   0.707107  \\
    0.40824829  &  -0.00000000  &   0.57735027 &    4  &  0.0 &   0.707107  \\
   -0.20412415  &   0.35355339  &   0.57735027 &    4  &  0.0 &   0.707107  \\
   -0.40824829  &  -0.70710678  &  -0.57735027 &    2  & 29.0 &   1.000000  \\
    0.81649658  &   0.00000000  &  -0.57735027 &    2  & 29.0 &   1.000000  \\
   -0.40824829  &   0.70710678  &  -0.57735027 &    2  & 29.0 &   1.000000  \\
    0.40824829  &  -0.70710678  &   0.57735027 &    4  &  0.0 &   1.000000  \\
   -0.81649658  &  -0.00000000  &   0.57735027 &    4  &  0.0 &   1.000000  \\
    0.40824829  &   0.70710678  &   0.57735027 &    4  &  0.0 &   1.000000  \\
 \end{tabular}
}
\bigskip

The example, presented here, has been obtained
from calculation for Cu (111)-surface  using decimation technique with 12 
layers slab: 3Cu/6Vac/3Cu. We are going to put one impurity atom on the Cu 
left-surface-layer, which has id-number 3 in the inputcard.  The number in the 
first row of the "scoef" file is the total number of atoms in the cluster, the 
first three columns are the $(z,y,z)$ coordinates, the id-numbers of the atoms 
(according to the inputcard)
are in the 4-th column, the atomic charges are in the 5-th column, 
the distances are in the last column. 

  Then set in the inputcard the correct name (say, "scoef") of the file I25, 
which you just have created, set IGREEN=1, ICC=1 and make one iteration of the 
'tb-kkr' program. You will obtain three files: "impurity.coefs", 
"intercell\underline{{ }}ref" and "green" (this last file is in the binary 
format, it has a huge size and contains the information
about the structural Green's function $G_{LL'}^{nn'}$ for all sites
$n$ and $n'$ of our cluster). Make the new directory for "impurity" 
calculation in you working directory and copy these 3 files to the new 
directory. 

\section{Impurity program}

To start impurity program, you should first prepare the initial potential.
The information for making such a potential you can find in 
the "impurity.coefs" file, and the utility program 'potutil.f' helps you to do 
that. This small FORTRAN file is usually situated in {\tt /util} subdirectory 
of the directory where your kkr-package is. You should compile this 'potutil.f'
file using simply {\tt g77} compiler, and should run the executable file
in your working directory. Copy the self-consistent potential of your 
referring system into the  working directory. After running the utility 
program, you should refer to the "impurity.coefs" file for the correct input 
information, which will be asked by the utility program. In our example of 
inpurity on the Cu (111)-surface, the "impurity.coefs" contains the following 
information:
\bigskip

{\small\tt
\hspace{0.5cm} Position of Impurity \hspace{1.9cm} Host Imp Shell \ Dist
\quad  Host id in Bulk

\noindent
\begin{tabular}{rrrrrrrr}
  0.00000000  &  0.00000000  & 0.00000000   & 2  & 1 &   1 & 0.000000  &  3 \\
  0.20412415  & -0.35355339  &-0.57735027   & 1  & 2 &   2 & 0.707107  &  2 \\
 -0.40824829  &  0.00000000  &-0.57735027   & 1  & 3 &   3 & 0.707107  &  2 \\
  0.20412415  &  0.35355339  &-0.57735027   & 1  & 4 &   4 & 0.707107  &  2 \\ 
  0.00000000  & -0.70710678  & 0.00000000   & 2  & 5 &   5 & 0.707107  &  3 \\
 -0.61237244  & -0.35355339  & 0.00000000   & 2  & 6 &   6 & 0.707107  &  3 \\
  0.61237244  & -0.35355339  & 0.00000000   & 2  & 7 &   7 & 0.707107  &  3 \\
 -0.61237244  &  0.35355339  & 0.00000000   & 2  & 8 &   8 & 0.707107  &  3 \\
  0.61237244  &  0.35355339  & 0.00000000   & 2  & 9 &   9 & 0.707107  &  3 \\
  0.00000000  &  0.70710678  & 0.00000000   & 2  & 10 & 10 & 0.707107  &  3 \\
 -0.20412415  & -0.35355339  & 0.57735027   & 3  & 11 & 11 & 0.707107  &  4 \\
  0.40824829  & -0.00000000  & 0.57735027   & 3  & 12 & 12 & 0.707107  &  4 \\
 -0.20412415  &  0.35355339  & 0.57735027   & 3  & 13 & 13 & 0.707107  &  4 \\
 -0.40824829  & -0.70710678  &-0.57735027   & 1  & 14 & 14 & 1.000000  &  2 \\
  0.81649658  &  0.00000000  &-0.57735027   & 1  & 15 & 15 & 1.000000  &  2 \\
 -0.40824829  &  0.70710678  &-0.57735027   & 1  & 16 & 16 & 1.000000  &  2 \\
  0.40824829  & -0.70710678  & 0.57735027   & 3  & 17 & 17 & 1.000000  &  4 \\
 -0.81649658  & -0.00000000  & 0.57735027   & 3  & 18 & 18 & 1.000000  &  4 \\
  0.40824829  &  0.70710678  & 0.57735027   & 3  & 19 & 19 & 1.000000  &  4 \\
\end{tabular}

304 \qquad  354

1

\noindent
Host order, no of host sites: \qquad 3

2 \qquad  3 \qquad  4
}
\bigskip

You see, that 3 host sites with id-numbers 2,\ 3 and 4 should be used
to produce the potential of the host, therefore the total number
of sites for the output potential is $19+3 = 22$ for this example.
You can then choose the number of spins (1 or 2) for your potential.
Then you input the order of atoms for the output potential 
in the following way: first must be the host sites in the order like 
they are given in the "impurity.coefs" file, e.g., 2,\ 3 and 4 for our 
example; then you input one by one the "host id-numbers in the bulk" (last 
column in the "impurity.coefs" file), e.g, 3,\ 2,\  2,\ 2,\ 3, \dots.  
(see the table above). The output file with initial potential has the name
"composed.potent". Then you can cut out from the "composed.potent" file
the potentials for some atoms belonging to the impurity cluster
and replace them by potentials of your impurities, which you can 
obtain either from 'voronoi' program (jellium potentials) or from
bulk calculation for impurity atoms.

 In your kkr-package directory you have the subdirectory usually called
{\tt /impurity} with impurity program. You should recompile (runnig {\tt make} 
file) the impurity program before a calculation of the new system, 
pointing out an attention to the parameters in "parameters.file".
The parameters you must care of before compilation are: {\tt NTREFD} -- 
the number host sites (=3 for our example), {\tt NATOMD} -- the number
of sites in the impurity cluster (=19 for our example), {\tt NSPIND} -- number 
of spins (1 or 2), {\tt IEMXD} must be larger, than the total number
{\tt NPT1 + NPT1 + NPT3 + NPOL} of points on the energy contour in the complex 
energy plane, {\tt NSEC} is the dimension of the structural Green's functon 
matrix (=304 for our example, the value of this patameter can be found in 
the "impurity.coefs" file), {\tt LMAXD} is $l_{\rm max}$ cutoff, and other 
parameters you usually do not have to change. The executable file name 
of the impurity program is usually "kkrimp.exe".

\end{document}
