\documentclass{llncs}

\usepackage{fancyhdr}
\renewcommand{\headrulewidth}{0pt}
\fancypagestyle{firstpage}{
  \fancyhf{}
  \fancyhead[C]{\normalsize{ORNL-FZJ -- Internal Draft Version -- Do NOT Distribute!}} 
}

\usepackage{todonotes}
\usepackage{amsmath}
\usepackage{amssymb}
\usepackage{graphicx}
\usepackage{booktabs}
\usepackage{units}
%\usepackage[svgpath=figs/]{svg}

\newcommand{\ttt}[1]{\texttt{#1}}
\newcommand{\um}[1]{_{\mathrm{#1}}}
\newcommand{\ket}[1]{ \left| #1 \right\rangle }
\newcommand{\ellmax}{\ell_{\mathrm{max}}}
 

\def\KKRnano{{KKRnano}}
\def\LSMS{{LSMS}}

\begin{document}

\title{Collaboration on Multiple Scattering Methods}

\author{ 
  Paul F.~Baumeister\inst{3} 
  \and
  Rudolf Zeller\inst{2}
  \and
  Marcel Bornemann\inst{2}
  \and
  Roman Kovacik\inst{2}
  \and
  Khorgolkhuu Odbadrakh\inst{4}
  \and
  Markus Eisenbach\inst{1}
}

\institute{ 
  Oak Ridge National Laboratory, Oak Ridge, Tennessee 37831, USA
  \and
  Peter-Gr\"unberg-Institut, Forschungszentrum J\"ulich, 52425
  J\"ulich, Germany
\and
  J\"ulich Supercomputing Centre, Forschungszentrum J\"ulich, 52425 J\"ulich, Germany
  \and
Joint Institute for Computational Sciences, University of Tennessee,
Oak Ridge, Tennessee 37831, USA
%   \and
%   IBM Deutschland Research \& Development GmbH, 71032 B\"{o}blingen, Germany 
%   \and 
%   IBM T.J.~Watson Research Center, Yorktown Heights, NY 10598, US
}

\maketitle
\thispagestyle{firstpage}

% ==================================================================================================
\begin{abstract}
Multiple scattering methods for electronic structure calculations
give natural access to linear scaling with system size.
Large scale simulations of realistic quantum effects are becoming increasingly important
for nanotechnology. Therefore, the developer teams of \LSMS{} at Oak Ridge National Laboratory
and \KKRnano{} at Forschungszentrum J\"ulich identify mathematical, algorithmic and technical issues
to improve their implementations of linear scaling density functional theory in close collaboration.
\end{abstract}
% ==================================================================================================

% ==================================================================================================
\section{Introduction} \label{section:intro}
% ==================================================================================================
This document describes the framework of a newly-established strong collaboration
between the Oak Ridge National Laboratory (ORNL) represented by Dr.~Markus Eisenbach
and Forschungszentrum J\"ulich GmbH (FZJ) represented by Dr.~Rudolf Zeller.
The aim of the collaboration is to promote the development of multiple scattering methods
for scalable high performance calculations in the framework of density functional theory
and related approaches towards the electronics and magentic structure of materials.
This collaboration will be strongly focussed onto the mutual long-term benefit of both research institutions, 
ORNL and FZJ, by sharing common experiences in the following three subtopics
\begin{itemize}
 \item Mathematical approaches towards specific physical problems  
 \item Stable numerical algorithms suitable for fast solutions
 \item Technical issues of porting the applications to current/future HPC platforms
\end{itemize}
Both sides are willing to reveal their knowledge on these topics in order to
contribute to fruitful discussions.


%% structure
In the following section \ref{section:apps}, a short description of the code packages is given as they are today.
Then common features and major differences between the two implementations are pointed out in section \ref{section:differ}.
We identify common goals for the future of the two applications in section \ref{section:common}
and highlight selected technical challenges that have to be overcome for reaching these goals in section \ref{section:tech}.
Finally, we summarize the contents of this collaboration in section \ref{section:summary}.

% ==================================================================================================
\section{Applications} \label{section:apps}
% ==================================================================================================

Both research institutions contribute experiences that are gained during the development
of their main applications in the field of multiple scattering method.
These implementations, the \LSMS{} code and \KKRnano{} from ORNL and FZJ, respectively, are described here.

% ..................................................................................................
\subsection{KKRnano} \label{section:kkrnano}
% ..................................................................................................
\KKRnano{} is an implementation of the Korringa-Kohn-Rostoker multiple scattering method \cite{PhysRevB.52.8807,PhysRevB.85.235103}
that allows to solve for the Green function of a system consisting of many atoms.
The decomposition of the simulation volume into Voronoi cells has shown to be a good
approach for a precise description of relevant physical quantities on radial grids within each cell.
Each Voronoi cell features a radial grid that originates at the cell center
which coincides with the position of the atomic nucleus unless we treat a vacuum cell.
If necessary, vacuum cells are added to calculations to improve the convergence with respect to the truncation of
the angular momenta, $\ellmax$, for various quantities, so that a calculation may involve more cells than atoms.
%% difficulty: how to find the best position of vacuum cells in an unordered geometry
The code performs self-consistency (SCF) iterations in order to converge the electron density
where in each SCF iteration, the new density is found as the imaginary part of the Green function.
The necessary energy integration is performed in a contour integration scheme that
samples points in the complex plane.
The energy-dependent Green function is found as the solution of the Dyson equation
that arises from the difference of the scattering at the actual potential and the potential 
of a reference system. As a reference system, \KKRnano{} uses a piece-wise constant potential
function that represents repulsive spheres located at the atoms sites.
This choice of reference system allows for a strong localization of the reference cluster
which is equivalent to a sparse, banded matrix. 
This allows for a linear scaling of the compute time for the application of the scattering operator
onto a trial vector that stands for one column of the Green function. 
The length of the trial vector and the number of columns that have to be solved, however, scale with
the number of visible atoms.
Range truncation allows to reduce the number of visible atoms to a fixed number 
of atoms inside a given radius that defines the interaction zone. This reduces the computational
complexity for the solution of the Dyson equation from quadratic to linear.
Parallel processing is implemented for distributed memory (MPI) and on a thread level (OpenMP).
Both paradigms may be activated at the same time so that huge levels of concurrency 
($> \,$1M) can be exploited to reduce the time-to-solution.

% ..................................................................................................
\subsection{LSMS} \label{section:lsms}
% ..................................................................................................

\LSMS{}\cite{PhysRevLett.75.2867}, like \KKRnano{}, implements the multiple
scattering KKR formalism in real space to calculate the Green's
function of condensed matter systems and it follows the same
fundamental decomposition of space into atomic cells and empty cells
as needed. The Dyson equation in \LSMS{} is solved with respect to the
free space reference system, which results in dense matrices for the
calculation of the scattering path matrix, which limits the size of
the local interaction zone to $O(100)$ scattering sites.

In addition of non relativistic and scalar relativistic
calculations, \LSMS{} allows the solution of the fully relativistic
Dirac equation for electron scattering. Thus all relativistic effects
including spin-orbit interactions are accounted for, which allows the
calculation of magnetocrystaline anisotropy energies and
Dzyaloshinskii-Moriya antisymmetric exchange interactions. The
energies for arbitrary non-collinear magnetic spin configurations can
be calculated using selfconsistently determined Lagrange multipliers
that constrain the local magnetic order. 

The implementation utilizes multiple levels of parallelism: We employ
distributed memory parallelism via MPI to parallelize over the atoms
in the system utilizing the natural real space domain decomposition
and minimal communication requirements in the multiple scattering part
of the calculation. On node, shared memory, parallelism is achieved
for both parallelization over atoms as well as over energy points on
the integration contour. Finally, the calculation of the multiple
scattering matrix uses GPU acceleration when available.  

An additional level of parallelism is provided by the capability to
perform Wang-Landau Monte-Carlo sampling of magnetic and chemical
order. This allows the first principles statistical physics
calculation of magnetic and ordering phase transitions. The current
implementation uses a master-slave approach of a single Monte-Carlo
process driving multiple \LSMS{} instances, thus extending the
scalability of this method by multiple orders of magnitude.

% ==================================================================================================
\section{Major Differences} \label{section:differ}
% ==================================================================================================

\paragraph{Radial grid representation}
While both codes use a representation on radial grids inside each of the Voronoi cells,
logarithmic radial grids allow to accurately represent cores states and other radial functions
close to the nucleus. \KKRnano{} stops the logarithmic grid at the muffin tin radius,
a radius chosen to touch the Voronoi cell hull from inside or even slightly smaller, and 
continues with a set of panels each hosting a uniform grid. This allows for a very accurate
description of the shape functions. \LSMS{}, in contrast, continues the logarithmic grid
to the outermost point of the cell with the advantage of a much simpler implementation.

\paragraph{Non-collinear spin treatment}
\LSMS{} can treat the direction of the magnetisation completely unconstrained
while \KKRnano{} only features a collinear treatment of spins, i.e. $\uparrow$ and $\downarrow$, 
or non-magentic calculations. However, the process of implementing non-collinear
magnetism alongside Spin-Orbit-Coupling (SOC) has recently begun.
MPI Parallelization over the collinear spins is possible.

%% scalar-relativistic single site (and core) solver vs. Dirac, SOC?
\paragraph{Core solver physics}
\KKRnano{} offers to choose between a scalar-relativistic and a non-relativistic solution for the core states 
and for the radial functions entering the single site expressions. 
\LSMS{}, in contrast, provides access to the fully relativistic
solution of the Dirac equation in addition to the non-relativistic and
scalar relativistic solutions for the single site scattering
solutions, thus naturally including spin-orbit interactions.
%
%\todo[inline]{how about the other theory levels in LSMS? SOC? SRA?}

%% direct vs. iterative inversion of the screened interaction
\paragraph{Direct vs.~iterative Dyson solvers}
The most time consuming part of \LSMS{} is a LU-decomposition necessary 
for the inversion that solves the Dyson equation. The LU-inversion in
\LSMS{} recursively applies the Schur complement of subblocks to
reduce the number of opperations needed to obtain the diagonal block
of the inverse.\cite{Eisenbach2016} 
%
% \todo[inline]{need more detail here}
%
\KKRnano{} constructs a screened KKR operator that features a short range and, hence, a small number
of non-zero block-bands in the matrix representation. This favors the usage of an iterative
inversion scheme. The transpose-free quasi-minimal-residual method (TFQMR) has been chosen here.

%% programming model: Fortran2008, F77, C++ on the outer layer
\paragraph{Programming model}
\KKRnano{} is written in modern Fortran (F2008) with some legacy parts in old F77 style.
\LSMS{} makes use of C++ (C++11) for the outer skeleton and Fortran routines
for the inner work kernel.
The GPU implementation of \LSMS{} utilizes both standard CUDA
libraries (cuBLAS) as well as CUDA-C implementations of individual kernels. 
%
%\todo[inline]{do you use CUDA Fortran or CUDA-C for the GPU kernels? or are there GPU-libraries}

%% GPU acceleration in LSMS
\paragraph{GPU acceleration}
\LSMS{} leverages the compute power of GPUs as shown on Titan, where a $8.6\times$ shorter
time-to-solution and $7.3\times$ lower energy-to-solution can be achieved.
A first GPU version for the Dyson solver of \KKRnano{} is currently work in progress.

%% parallelization levels
\paragraph{Parallelization levels}
\KKRnano{} allows to distribute cells, collinear spins and energy points over MPI processes,
while the latter is only efficient to a limited degree due to the strong imbalance of the 
Matsubara poles compared to other energy points. \LSMS{} only
parallelizes over atomic cells at the MPI level and over cells and
energy points at the OpenMP level. The focus on non-collinear
magnetism and relativity prevents a simple parallelization over spin channels.

% ==================================================================================================
\section{Common Goals} \label{section:common}
% ==================================================================================================

%\todo[inline]{check for consistency and add missing topics}
The two teams identified the following problems in physics as target systems to which \LSMS{} and \KKRnano{}
are good candidates to be applied to
\begin{itemize}
 \item high entropy alloys
 \item huge systems with disorder
 \item defects in ordered structures
 \item full functional systems (e.g.~transistors)
 \item magnetic materials (e.g.~Skyrmions)
 \item structural materials (e.g.~metallic glasses)
\end{itemize}

In order to describe the physics correctly, \KKRnano{} is missing an implementation of non-collinear magnetism.


The following requirements on the results are essential to a successful application that gains visibility and impact in the community
\begin{itemize}
 \item accuracy
 \item reliability
\end{itemize}
whereas the requirement onto the implementations are
\begin{itemize}
 \item robustness
 \item efficiency
 \item resource utilization 
\end{itemize}
The latter two usually are addressed together when efforts are made to port the application to a new compute platform.
Often, the porting proceeds in two steps: Get the code functionality correct first, then, increase its efficiency on
the given architecture. Here, experiences of the \LSMS{} group could be transferred to the \KKRnano{} group who just
started to experiment with a GPU-accelerated solver kernel.

% ==================================================================================================
\section{Technical Challenges} \label{section:tech}
% ==================================================================================================

Here, a list of challenges is given that the developer teams of both applications, \LSMS{} and \KKRnano{}, are
facing. The aim of collecting these items is to give an overview of what needs to be done in the future
and to align efforts so that solutions found in one team may be propagated into the other one.
%\todo[inline]{add missing callenges}

% ..................................................................................................
\subsection{Fast Poisson solver for distributed charges} %\label{section:}
% ..................................................................................................
In each self-consistency iteration the Poisson equation needs to be solved to find the
electrostatic potential contribution. The charge density distribution (right hand side of the Poisson equation)
is given as a multi-center expansion on radial grids. The density is limited in its spatial extent
by the Voronoi-cell construction that enters the expressions as so-called \emph{shape functions}.
\KKRnano{} employs an Ewald summation technique to separate the long-range and short-range parts of the electrostatic interaction.
The long range parts are treated with a small number of lattice vectors in reciprocal space. The short-range parts
are found by summation over periodic images in real space.
The scaling of this solver quadratic %\todo[inline]{check again} 
and, hence, the algorithm is not suitable
for the treatment of huge systems where linear scaling is required in all components of the method.

The requirements for a Poisson solver are stability, accuracy, complexity and speed.
Since these requirements are fullfilled by the Fast Multipole Method (FMM), 
an investigation of how the FMM could be applied to the
electrostatic problem in KKR codes would shine light into this issue.
In particular it remains to be understood if the delocalized charges require a more
sophisticated treatment than usual FMM for point charges.

% ..................................................................................................
\subsection{Placement of empty cells in structural relaxation} %\label{section:}
% ..................................................................................................

Investigations of defect structures are often related to irregular
shapes of the Voronoi cell. Therefore, vacuum/empty cells are inserted that do not carry an atomic nucleus
but an expansion center for a radial grid.
The exact position of the empty cell's center is in principle arbitrary, however, the quality of results
depends on the quality of description of densities and potentials inside the cells. Therefore,
a strongly irregular Voronoi cell is not likely to describe these quantities well with a truncated $\ell$-expansion.
In particular during structural relaxation, the reconstruction of the geometry
asks for an automated method for the placement of empty cells.

% ..................................................................................................
\subsection{Reliable calculations of interatomic forces} %\label{section:}
% ..................................................................................................

The calculation of reliable forces in a full potential scheme is
required for structural relaxation of atomic positions and the
calculation of phononic properties. To achieve this goal, the robust  accuracy
of a number of different algorithms needs to be ensured. In
particular, energy variations due to changes in core state  boudary
conditions and errors due to chnging atomic cell boundary
volumes need to be addressed when atomc positions change.

% ..................................................................................................
\subsection{Self-consistency convergence acceleration} %\label{section:}
% ..................................................................................................

The simulation of larger and larger systems has shown that the usual recipe for the
convergence of the self-consistency cycles are no longer feasable without accepting a critical
slowing down with growing system size. 
This, however, contradicts the principles of an $\mathcal O(N)$-method.
New ways have to be identified in order to accelerate the SCF convergence, especially in systems
that feature electronic states close to the Fermi level, and, hence, suffer from charge sloshing.


% ==================================================================================================
\section{Conclusions}\label{section:summary}
% ==================================================================================================

The two KKR-based DFT applications \LSMS{} and \KKRnano{} identified common aims and similar target problems.
A list of potential challenges in which a close collaboration could lead to benefits on both sides has been assembled. 
Among algorithmical issues, three items have been highlighted in particular: 
\begin{itemize}
 \item Acceleration of the self-consistency loop,
 \item placement of empty cells and
 \item a scalable Poisson solver.
\end{itemize}
Furthermore, the collaboration between the developer teams of 
\LSMS{} from ORNL 
and
\KKRnano{} from FZJ
is meant to provide a platform for fast exchange of experiences on
bringing the applications to suitable computer architectures and running them in high parallelism.

%   ==================================================================================================
\section*{Acknowledgments}
% ==================================================================================================
We thank Jack Wells and Tjerk Straatsma from ORNL and Dirk Pleiter and Stefan Bl\"ugel from FZJ 
for their efforts making this collaboration possible.

% ==================================================================================================
\bibliographystyle{plain} \bibliography{LSMS-KKRnano-Common-Interests} %% LSMS-KKRnano-Common-Interests.bib file
% ==================================================================================================
\end{document}
