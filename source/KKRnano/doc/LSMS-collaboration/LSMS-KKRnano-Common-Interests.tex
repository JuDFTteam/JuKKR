\documentclass{llncs}

\usepackage{fancyhdr}
\renewcommand{\headrulewidth}{0pt}
\fancypagestyle{firstpage}{
  \fancyhf{}
  \fancyhead[C]{\normalsize{UCHPC 2016 -- Draft Version -- Do NOT Distribute!}} 
}

\usepackage{todonotes}
\usepackage{amsmath}
\usepackage{amssymb}
\usepackage{graphicx}
\usepackage{booktabs}
\usepackage{units}
%\usepackage[svgpath=figs/]{svg}

\newcommand{\ttt}[1]{\texttt{#1}}
\newcommand{\um}[1]{_{\mathrm{#1}}}
\newcommand{\ket}[1]{ \left| #1 \right\rangle }
\newcommand{\ellmax}{\ell_{\mathrm{max}}}
 

\def\KKRnano{{KKRnano}}
\def\LSMS{{LSMS}}

\begin{document}

\title{Collaboration on Multiple Scattering Methods}

\author{ 
  Paul F.~Baumeister\inst{3} 
  \and
  Rudolf Zeller\inst{2}
  \and
  Markus Eisenbach\inst{1}
  \and
  Od ??\inst{1}
}

\institute{ 
  Oak Ridge National Laboratory, Tennessee, US
  \and
  Peter-Gr\"unberg-Institut, Forschungszentrum J\"ulich, 52425 J\"ulich, Germany
  J\"ulich Supercomputing Centre, Forschungszentrum J\"ulich, 52425 J\"ulich, Germany
  \and
%   \and
%   IBM Deutschland Research \& Development GmbH, 71032 B\"{o}blingen, Germany 
%   \and 
%   IBM T.J.~Watson Research Center, Yorktown Heights, NY 10598, US
}

\maketitle
\thispagestyle{firstpage}

% ==================================================================================================
\begin{abstract}
  ABSTRACT TO BE WRITTEN LAST
\end{abstract}
% ==================================================================================================

% ==================================================================================================
\section{Introduction} \label{section:intro}
% ==================================================================================================
This document describes the framework of a newly-established strong collaboration
between the Oak Ridge National Laboratory (ORNL) represented by Dr. Markus Eisenbach
and Forschungszentrum J\"ulich GmbH (FZJ) represented by Dr. Rudolf Zeller.
A Memorandum of Understanding between the two research facilities has been set in place 20?? 
that fosters such direct collaborations on a scientific level.
The aim of the collaboration is to promote the development of multiple scattering methods
for scalable high performance calculations in the framework of density functional theory
and related approaches towards the electronics and magentic structure of materials.
This collaboration will be strongly focussed onto the mutual long-term benefit of both research institutions, 
ORNL and FZJ, by sharing common experiences in the following three subtopics
\begin{itemize}
 \item Mathematical approaches towards specific physical problems  
 \item Stable numerical algorithms suitable for fast solutions
 \item Technical issues of porting the applications to current and future HPC platforms
\end{itemize}
Both sides are willing to reveal their knowledge on these topics in order to
contribute to fruitful discussions.


%% structure
In the following section \ref{section:apps}, a short description of the code packages is given as they are today.
Then common features and mayor differences between the two implementations are pointed out in section \ref{section:differ}.
We identify common goals for the future of the two applications in section \ref{section:common}
and highlight selected technical challenges that have to be overcome for reaching these goals in section \ref{section:tech}.
Finally, we summarize the aim of this collaboration in section \ref{section:summary}.

% ==================================================================================================
\section{Applications} \label{section:apps}
% ==================================================================================================

Both research institutions contribute experiences that are gained during the development
of their main applications in the field of multiple scattering method.
These implementations, the \LSMS code and \KKRnano from ORNL and FZJ, respectively, are described here.

% ..................................................................................................
\paragraph{KKRnano} \label{section:kkrnano}
% ..................................................................................................
is an implementation of the Korringa-Kohn-Rostoker multiple scattering method \cite{korringa, kohn-rostoker}
that allows to solve for the Green function of a system consisting of many atoms.
The decomposition of the simulation volumen into Voronoi cells has shown to be a good
approach for a precise description of relevant phyical quantities on radial grids.
Each Voronoi cell features a radial grid that originates at the cell center
which coincides with the position of the atomic nucleus unless we treat a vacuum cell.
Vacuum cells are added to calculations to improve the convergence with respect to the truncation of
the angular momenta, $\ellmax$, for various quantities, so that a calculation may involve more cells than atoms.
%% difficulty: how to find the best position of vacuum cells in an unordered geometry
The code performs self-consistency (SCF) iterations in order to converge the electron density
where in each SCF iteration, the new density is found as the imaginary part of the Green function.
The necessary energy integration is performed in a contour integration scheme that
samples points in the complex plane.
The energy-dependent Green function is found as the solution of the Dyson equation
that arises from the difference of the scattering at the actual potential and the potential 
of a reference system. As a reference system, \KKRnano uses a piece-wise constant potential
function that represents repulsive spheres located at the atoms sides.
This choice of reference system allows for a strong localization of the reference cluster
which is equivalent to a sparse, banded matrix. 
This allows for a linear scaling of the compute time for the application of the scattering operator
onto a trial vector that stands for one column of the Green function. 
The length of the trial vector and the number of columns that have be solved, however, scale with
the number of visible atoms.
Range truncation allows to reduce the number of visible atoms to a fix number 
of atoms inside a given radius that defines the interaction zone. This reduces the computation
complexity for the solution of the Dyson equation from quadratic to linear.
Parallel processing is implemented for distributed memory (MPI) and on a thread level (OpenMP).
Both paradigms may be activated at the same time so that huge levels of concurrency 
(> 1M) can be exploited to reduce the time-to-solution.

% ..................................................................................................
\paragraph{LSMS} \label{section:lsms}
% ..................................................................................................
\todo[inline]{to be written}


% ==================================================================================================
\section{Mayor Differences} \label{section:differ}
% ==================================================================================================
%%radial grid

% ==================================================================================================
\section{Common Goals} \label{section:common}
% ==================================================================================================

% ==================================================================================================
\section{Technical Challenges} \label{section:tech}
% ==================================================================================================

% ==================================================================================================
\section{Conclusions}\label{section:summary}
% ==================================================================================================

%   ==================================================================================================
\section*{Acknowledgments}
% ==================================================================================================
We thank Jack Wells and Tjerk Straatsma from ORNL and Dirk Pleiter from FZJ 
for their efforts that made this collaboration possible.

% ==================================================================================================
\bibliographystyle{plain} \bibliography{KKR} %% KKR.bib file
% ==================================================================================================
\end{document}
